% !TEX encoding = System

%%\documentclass[thesismargins, thesislinespacing, twoside, draft, upjsfrontpage]{rnthesis}
\documentclass{rnthesis}
\usepackage[slovak]{babel}
\usepackage[T1]{fontenc}
\usepackage[cp1250]{inputenc}
%\usepackage[utf8]{inputenc}
\usepackage{rnt-pic}
\usepackage{rnt-thm}
\usepackage{amsfonts}
\usepackage{amsmath}
\usepackage{color}
\usepackage{listings}
\usepackage{hyphenat}
\usepackage{multirow}
\usepackage{graphicx}

\hyphenation{funk-cií}

\lstset{language=[Sharp]C,
showspaces=false,
showtabs=false,
breaklines=true,
%basicstyle=\tiny,
basicstyle=\fontsize{9}{13}\selectfont\ttfamily,
numbersep=5pt,
tabsize=2,
showstringspaces=false,
breakatwhitespace=true,
escapeinside={(*@}{@*)},
commentstyle=\color{greencomments},
keywordstyle=\color{bluekeywords}\bfseries,
stringstyle=\color{redstrings},
%basicstyle=\ttfamily
morecomment=[l]{//},
morecomment=[s]{/*}{*/},
morekeywords={ abstract, event, new, struct,
as, explicit, null, switch,
base, extern, object, this,
bool, false, operator, throw,
break, finally, out, true,
byte, fixed, override, try,
case, float, params, typeof,
catch, for, private, uint,
char, foreach, protected, ulong,
checked, goto, public, unchecked,
class, if, readonly, unsafe,
const, implicit, ref, ushort,
continue, in, return, using,
decimal, int, sbyte, virtual,
default, interface, sealed, volatile,
delegate, internal, short, void,
do, is, sizeof, while,
double, lock, stackalloc,
else, long, static,
enum, namespace, string, partial, get, set, var},
}
%\usepackage{xcolor}

%\lstdefinestyle{sharpc}{language=[Sharp]C, frame=lr, rulecolor=\color{blue!80!black}}
%----------------------------------------------------------------------------------------
\newcommand{\real}{$\mathbb{R}$}
\newcommand{\uinteger}{$\mathbb{N} \cup \{0\}$} 
\newcommand{\nat}{$\mathbb{N} \cup \{0\}$} 
\newcommand{\integer}{$\mathbb{Z}$}
\newcommand{\mcomma}{\mbox{,}}
\newcommand{\FEL}{\frac{1}{2}}
\newcommand*{\mathcolor}{}
\def\mathcolor#1#{\mathcoloraux{#1}}
\newcommand*{\mathcoloraux}[3]{%
  \protect\leavevmode
  \begingroup
    \color#1{#2}#3%
  \endgroup
}
\def\pd#1#2{\frac{\partial #1}{\partial #2}}
\def\pdd#1#2#3{\frac{\partial #1}{\partial #2\partial#3}}

\def\TP#1{\mathit{#1}}
\def\FN#1{\text{#1}}
\def\uv#1{„{#1}“}
\definecolor{basis1}{rgb}{0.64,0.29,0.64}
\definecolor{basis2}{rgb}{0,0.64,0.91}
\definecolor{basis3}{rgb}{0.13,0.69,0.3}
\definecolor{basis4}{rgb}{0.93,0.11,0.14}
\definecolor{basis1a}{rgb}{0.90,0.6,0.90}
\definecolor{basis2a}{rgb}{0.2,0.80,1}
\definecolor{basis3a}{rgb}{0.3,0.9,0.5}
\definecolor{basis4a}{rgb}{1,0.7,0.7}

%\definecolor{dkgreen}{rgb}{0,0.6,0}
%\definecolor{gray}{rgb}{0.5,0.5,0.5}
%\definecolor{mauve}{rgb}{0.64,0.08,0.08}

\definecolor{bluekeywords}{rgb}{0.13,0.13,1}
\definecolor{greencomments}{rgb}{0,0.5,0}
\definecolor{redstrings}{rgb}{0.64,0.08,0.08}
%----------------------------------------------------------------------------------------
\catcode`@=11
\def\xmatrix#1#2{\null\,\vcenter{\normalbaselines\m@th
    \ialign{\hfil$##$\hfil&&\hskip#1\hfil$##$\hfil\crcr
      \mathstrut\crcr\noalign{\kern-\baselineskip}
      #2\crcr\mathstrut\crcr\noalign{\kern-\baselineskip}}}\,}
\catcode`@=12

\title{Algoritmizácia, paralelizácia a implementácia splajn modelov}
%\subtitle{Algoritmizácia, paralelizácia a implementácia splajn modelov}
\author{Bc. Viliam Kačala}
\typprace{Analýza a návrh riešenia}
\rok{2015}
\miesto{Košice}
%\podakovanie{
% Rád by som poďakoval vedúcemu záverečnej
% práce doc. RNDr. Csabovi Törökovi, CSc.
% za cenné pripomienky, odborné vedenie a obetavosť počas
% tvorby práce. Taktiež by som sa rád poďakoval 
% RNDr. Lukášovi Miňovi za cenné rady a odbornú
% pomoc počas tvorby aplikačnej časti tejto práce.
%} 
\veduci{doc. RNDr. Csaba Török, CSc..}
\konzultant{RNDr. Lukáš Miňo}
\pracovisko{Ústav informatiky}

\abstract{
	Splines are important part of computer graphics.
	It is a mathematical model of the surface, which is for the "best connection" of any finite set of points.
	The term "best connection" in this case means smooth, easily calculable mathematical surface with minimal curvature.
	Use of splines in graphics varies from large variety of CAD applications, statistics or in data analysing.
	Splines exist in many forms, whether in the form of curves in the plane, a variety of three-dimensional bodies, etc. ..
	This work aims to desig, analyze and implement new algorithm for counting and generating splines bicubic clamped interpolation in three-dimensional space.
}

\abstrakt{
	Splajny sú dôležitá súčasť počítačovej grafiky.
	Jedná sa o matematický model plochy, ktorá slúži na čo „najlepšie spojenie“ konečnej množiny bodov. 
	Termín „najlepšie spojenie“ v našom prípade znamená hladká, matematicky ľahko vyjadriteľná plocha s čo najmenším zakrivením.
	Využitie splajnov v grafike je veľmi široké od rôznych CAD aplikácií, v štatistike, alebo v analýze dát.
	Splajny existujú v mnohých formách, či už vo forme krivky v rovine, rôznych trojrozmerných telies, atď..
	Táto práca si dáva za cieľ navrhnúť, analyzovať a implementovať nový algoritmus pre bikubickú interpoláciu v trojrozmernom priestore.
}

\bibliographystyle{alpha}

\begin{document}
%\lstset{style=sharpc}
%\lstset{frame=tb,
%  language=[Sharp]C,
%  aboveskip=3mm,
%  belowskip=3mm,
%  showstringspaces=false,
%  columns=flexible,
%  basicstyle={\small\ttfamily},
%  numbers=none,
%  numberstyle=\tiny\color{gray},
%  keywordstyle=\color{blue},
%  commentstyle=\color{dkgreen},
% stringstyle=\color{mauve},
%  breaklines=true,
%  breakatwhitespace=true
%  tabsize=3
%}

\maketitle
%\newpage
%\tableofcontents
%\newpage
%==================================================================================================================================================================================
\section*{Úvod}

Témou mojej diplomovej práce sú priestorové splajn povrchy, kde naším cieľom je preskúmať nové poznatky o splajnoch, na ktorých istý čas pracuje vedúci tejto práce doc. RNDr. Csaba Török, CSc.
Výsledkom tejto práce je okrem návrhu tejto novej metódy výpočtu aj porovnanie so známymi metódami a implementácia aplikácie na vizuálne porovnávanie jednotlivých metód.
Technológia v ktorej budeme tieto poznatky implementovať je Microsoft Silverlight. Jedná o veľmi schopný nástroj na tvorbu webových aplikácii s plnou podporou hardvérovej akcelerácie užívateľského prostredia.
Výhodou tohto frameworku je, kedže beží na platforme Microsoft .NET, možnosť jednoduchej portácie na desktopovú prípadne mobilnú aplikáciu. Štruktúra práce je nasledovná.
\begin{itemize}
\item{\textbf{Splajn}}
Definícia pojmu splajn.

%V úvode si vysvetlíme technológiu Silverlight a základy práce grafiky.
\item{\textbf{De Boorova interpolácia}}
V tejto časti si vysvetlíme bikubickú splajn interpoláciu podľa Carla de Boora. 

\item{\textbf{Nový postup generovania bikubických splajnov}}
Ukážka modifikovaného postupu pre kubické splajny a jeho rozšírenie pre bikubické splajny.

\item{\textbf{Zrýchlenie}}
Očakávané zrýchlenie výpočtov novým algoritmom.


\item{\textbf{Implementácia}}

Podrobnosti implementácie v MS Silverlight.
\end{itemize}
Formálne tézy diplomovej práce sú:

\begin{itemize}

	\item
	Analýza modelov interpolačných splajn-povrchov a podpory paralelného programovania. .NET Framework
	\item
	Algoritmizácia, paralelizáciaa vizualizácia splajn-povrchov.
	\item
	Implementácia paralelných algoritmov vybraných modelov splajn-povrchov pomocou viacvláknových API pre multiprocesorové systémy a ich porovnanie.
\end{itemize}

\section*{Splajn}

V našej práci pracujeme s hermitovskými splajnami, ktoré sú štandardne triedy $C^1$, teda splajnami, ktorých prvé derivácie v uzloch sa rovnajú. V našej práci však budeme uvažovať hermitovské splajny triedy $C^2$, teda splajny pri ktorých sa v uzloch rovnajú navyše aj derivácie druhého rádu.

\obrazok{mexHatWKnots.png}{0.7}{Funkcia $sin(\sqrt{x^2+y^2})$ aproximovaná bikubickým splajnom.}
Formálnejšie si popíšme základné pojmy s ktorými budem v tomto článku pracovať. 

\oznac{
	Postupnosť $(a_0, a_1, \dots)$ nazveme \pojem{rovnomernou} ak pre ľubovoľné $i$,$j$ z $\{0, 1, \dots\}$ platí $a_{j+1} - a_j = a_{i+1} - a_i$.
}

\oznac{
Nech $f: \mathbb{R} \times \mathbb{R} \to \mathbb{R}$ je nejaká funkcia a $I$,$J$ sú z \uinteger. Ďalej nech $(u_0, \dots,u_{I+1})$ a $(v_0, \dots,v_{J+1})$ sú dve rovnomerné usporiadané konečné postupnosti reálnych čísel. Označme:
\begin{itemize}
	\item
	Dvojice $(u_i, v_j)$ nazveme \pojem{uzly}.
	\item
	Uzly $(u_0, v_0)$, $(u_{I+1}, v_0)$, $(u_0, v_{J+1})$ a $(u_{I+1}, v_{J+1})$ nazveme \pojem{rohové uzly}.
	\item
	$z_{i,j} = f(u_i, v_j)$.
	\item
	$dx_{i,j} = \pd{f(u_i, v_j)}{x}$.
	\item
	$dy_{i,j} = \pd{f(u_i, v_j)}{y}$.
	\item
	$dxy_{i,j} = \pdd{f(u_i, v_j)}{x}{y}$.
	\item
	$\Delta x = u_i - u_{i-1}$ a $\Delta y = v_j - v_{j-1}$ pre ľubovoľné $i \in  \{1, \dots, I+1\}$, resp. $j \in  \{1, \dots, J+1\}$.
\end{itemize}
}

Úlohou je na základe vstupných uzlov nájsť \pojem{hladkú}, po častiach definovanú funkciu $F: [u_0, u_{I+1}] \times [v_0, v_{J+1}] \to \mathbb{R}$ z so spojitými deriváciami prvého a druhého rádu takú, že pre každé $i \in  {0, \dots, I+1}$ a $j \in  {0, \dots, J+1}$ platí $F(u_i, v_j) = z_{i,j}$. Funkcia $F$ predstavuje splajn, pričom jej jednotlivé časti nazveme segmenty. 
Teraz si môžeme splajn zadefinovať formálne.

\begin{df}
	Nech $I$ a $J$ sú prirodzené čísla, $\mathbf{u} = (u_0, \dots,u_{I+1})$ a $\mathbf{v} = (v_0, \dots,v_{J+1})$ sú reálne čísla.
	\begin{itemize}
		\item
		$I \ge 0$ a $J \ge 0$,
		\item 
		Postupnosti $\mathbf{u}$ a $\mathbf{v}$ sú rovnomerne rastúce.
		\item
		Nech pre každé $i$ z $\{0, \dots, I\}$ a $j$ z $\{0, \dots, J\}$ je $S_{i,j}$ bikubická polynomická funkcia z  2-rozmerného intervalu $[u_{i}, u_{i+1}] \times [v_{j}, v_{j+1}]$ do \real.
	\end{itemize}
	%\newpage
	%\colorbox{yellow}{Spline}
	Definujme funkciu $S_{U,V}$ z 2-rozmerného intervalu $[u_{0}, u_{I+1}] \times [v_{0}, v_{J+1}]$ do $\mathbb{R}$ vzťahom:
	%$$\text{Spline}(x) = P_1(x), v_0 \le x < v_1\mbox{,}$$
	%$$\text{Spline}(x) = P_2(x), v_1 \le x < v_2\mbox{,}$$
	%$$\vdots$$
	%$$\text{Spline}(x) = P_k(x), v_{k-1} \le x \le v_k\mbox{,}$$
	%\begin{equation}
	\begin{equation}
	S_{\mathbf{u},\mathbf{v}}(x,y) = \begin{cases}
	S_{0,0}(x,y) & \text{pre } (x, y) \in [u_{0}, u_{1}] \times [v_{0}, v_{1}] \text{,}\\
	S_{0,1}(x,y) & \text{pre } (x, y) \in [u_{0}, u_{1}] \times [v_{1}, v_{2}] \text{,}\\
	\vdots\\
	S_{0,J}(x,y) & \text{pre } (x, y) \in [u_{0}, u_{1}] \times [v_{J}, v_{J+1}] \text{,}\\
	S_{1,0}(x,y) & \text{pre } (x, y) \in [u_{1}, u_{2}] \times [v_{0}, v_{1}] \text{,}\\
	\vdots\\
	S_{1,J}(x,y) & \text{pre } (x, y) \in [u_{1}, u_{2}] \times [v_{J}, v_{J+1}] \text{,}\\
	\vdots\\
	S_{I,0}(x,y) & \text{pre } (x, y) \in [u_{I}, u_{I+1}] \times [v_{0}, v_{1}] \text{,}\\
	\vdots\\
	S_{I,J}(x,y) & \text{pre } (x, y) \in [u_{I}, u_{I+1}] \times [v_{J}, v_{J+1}] \text{,}
	\end{cases}
	\end{equation}
	pričom všetky susedné segmenty $S_{i,j}$ majú derivácie prvého a druhého rádu v uzloch navzájom rovné.
\end{df}

\obrazok{fseg.png}{0.5}{\label{pic:fseg}Ukážka uzlov pre štvorsegmentový splajn.}

Ak $i$ je z  $\{0, \dots, I\}$ a $j$ je z $\{0, \dots, J\}$, tak štvorice uzlov $(u_i, v_j)$, $(u_i, v_{j+1})$, $(u_{i+1}, v_j)$ a $(u_{i+1}, v_{j+1}$ tvoria obdĺžnikový úsek nad ktorým sa nachádza splajnový segment. Každý segment $S_{i,j}$ je bikubická funkcia z $[u_i, u_{i+1}] \times [v_j, v_{j+1}]$ do $\mathbb{R}$. Výsledná funkcia $F$ teda vznikne zjednotením segmentov $S_{i,j}$. Na vygenerovanie každého segmentu potrebujeme štyri uzly, a pre každý uzol príslušne hodnoty $z$, $dx$, $dy$ a $dxy$. Tieto hodnoty vieme získať buď priamo z interpolovanej funkcie $f$, alebo ich vypočítať iným spôsobom.

\section*{De Boorova interpolácia}

Nech sú dané dané hodnoty $(u_0$, ...,$u_{I+1})$ a $(v_0$, ...,$v_{J+1})$, kde $I$,$J$ sú z \uinteger, pričom chceme interpolovať funkciu $S$ na $[u_0, u_I] \times [v_0, v_J]$. Výsledný splajn bude tvorený $(I+1) \cdot (J+1)$ segmentami. Ako bolo spomenuté, každý segment potrebuje štyri uzly a hodnoty $z$, $dx$, $dy$ a $dxy$. Pri de Boorovej interpolácii nám treba poznať tieto hodnoty:
\begin{itemize}
	\item
	$z_{i,j}$ pre $i \in \{0, \dots, I+1\}$, $j \in \{0, \dots, J+1\}$.
	\item
	$dx_{i,j}$ pre $i \in \{0, I+1\}$, $j \in \{0, \dots, J+1\}$.
	\item
	$dy_{i,j}$ pre $i \in \{0, \dots, I+1\}$, $j \in \{0, J+1\}$.
	\item
	$dxy_{i,j}$ pre $i \in \{0, I+1\}$, $j \in \{0, J+1\}$.
\end{itemize}

Na príklade uzlov z obrázka \ref{pic:fseg} potrebujeme poznať hodnoty takto:
\begin{itemize}
	\item
	$z_{0,2}$, $z_{1,2}$, $z_{2,2}$,\newline
	$z_{0,1}$, $z_{1,1}$, $z_{2,1}$,\newline
	$z_{0,0}$, $z_{1,0}$, $z_{2,0}$,\newline  
	\item
	$dx_{0,2}$, \phantom{$dx_{1,2}$}, $dx_{2,2}$,\newline
	$dx_{0,1}$, \phantom{$dx_{1,1}$}, $dx_{2,1}$,\newline
	$dx_{0,0}$, \phantom{$dx_{1,0}$}, $dx_{2,0}$,\newline 
	\item
	$dy_{0,2}$, $dy_{1,2}$, $dy_{2,2}$,\newline
	\phantom{$dy_{0,1}$}, \phantom{$dy_{1,1}$}, \phantom{$dy_{2,1}$}\newline
	$dy_{0,0}$, $dy_{1,0}$, $dy_{2,0}$,\newline
	\item
	$dxy_{0,2}$, \phantom{$dxy_{1,2}$}, $dxy_{2,2}$,\newline
	\phantom{$dxy_{0,1}$}, \phantom{$dxy_{1,1}$}, \phantom{$dxy_{2,1}$},\newline
	$dxy_{0,0}$, \phantom{$dxy_{1,0}$}, $dxy_{2,0}$,\newline  
\end{itemize}

Zvyšné hodnoty $z$, $dx$, $dy$ a $dxy$ vieme jednoznačne vypočítať pomocou $2(I+1) + J + 6$ lineárnych sústav s celkovo $3(I+1)(J+1) + I + J + 2$ rovnicami:
Nižšie uvádzame modelové rovnice, pomocou ktorých sú zostrojené tieto sústavy lineárnych rovníc.
\newline
Pre $j \in \{0, \dots, J+1\}$
\begin{multline} \label{eq:deboor1}
dx_{i+1,j} + 4dx_{i,j} + dx_{i+1,j}
= \frac{3}{\Delta x}(z_{i+1,j} - z_{i-1,j}) \text{, }\\ i \in \{1, \dots, I\}
\end{multline}
Pre $j \in \{0, J+1\}$
\begin{multline} \label{eq:deboor2}
dxy_{i+1,j} + 4dxy_{i,j} + dxy_{i+1,j}
= \frac{3}{\Delta x}(dy_{i+1,j} - dy_{i-1,j}) \text{, }\\ i \in \{1, \dots, I\}
\end{multline}
Pre $i \in \{0, \dots, I+1\}$
\begin{multline} \label{eq:deboor3}
dy_{i,j+1} + 4dy_{i,j} + dy_{i,j-1}
= \frac{3}{\Delta y}(z_{i,j+1} - z_{i,j-1}) \text{, }\\ j \in \{1, \dots, J\}
\end{multline}
Pre $i \in \{0, \dots, I+1\}$
\begin{multline} \label{eq:deboor4}
dxy_{i,j+1} + 4dxy_{i,j} + dxy_{i,j-1}
= \frac{3}{\Delta y}(dx_{i,j+1} - dx_{i,j-1}) \text{, }\\ i \in \{1, \dots, J\}
\end{multline}
\newline
Každá z týchto sústav má takýto maticový tvar:

\begin{equation} \label{eq:deboorM}
\begin{pmatrix}
4 & 1 & 0 & \hdots & 0 & 0\\
1 & 4 & 1 & \hdots & 0 & 0\\
0 & 1 & 4 & \hdots & 0 & 0\\
\vdots & \vdots & \vdots & \ddots & \vdots & \vdots\\
0 & 0 & 0 & \hdots & 4 & 1\\
0 & 0 & 0 & \hdots & 1 & 4\\
\end{pmatrix}
\cdot
\begin{pmatrix}
D_{1}\\
D_{2}\\
D_{3}\\
\vdots \\
D_{N-1}\\
D_{N}\\
\end{pmatrix}
=
\begin{pmatrix}
\frac{3}{\Delta}(Y_{2} - Y_{0}) - D_0\\
\frac{3}{\Delta}(Y_{3} - Y_{1})\\
\frac{3}{\Delta}(Y_{4} - Y_{2})\\
\vdots \\
\frac{3}{\Delta}(Y_{N} - Y_{N-2})\\
\frac{3}{\Delta}(Y_{N+1} - Y_{N-4}) - D_{N+1}\\
\end{pmatrix} \text{,}
\end{equation}
kde podľa toho o ktorú z modelových rovníc sa jedná, hodnoty N, D a Y zadávame následovne. Nech $k$ z ${1, \dots, K-1}$:
\begin{itemize}
	\item
	$N = I$, $\Delta = \Delta x$,  $D_k = dx_{k,j}$ a $Y_k = z_{k,j}$, pre rovnicu \ref{eq:deboor1}.
	\item
	$N = I$, $\Delta = \Delta x$,  $D_k = dxy_{k,j}$ a $Y_k = dy_{k,j}$, pre rovnicu \ref{eq:deboor2}.
	\item
	$N = J$, $\Delta = \Delta y$,  $D_k = dy_{i,k}$ a $Y_k = z_{i,k}$, pre rovnicu \ref{eq:deboor3}.
	\item
	$N = J$, $\Delta = \Delta y$,  $D_k = dxy_{i,k}$ a $Y_k = dx_{i,k}$, pre rovnicu \ref{eq:deboor3}.
\end{itemize}

Tento splajn jednoznačne interpoluje funkciu $S$.

\pozn{
De Boorova interpolácia vo všeobecnosti nepredpokladá len rovnomerne rastúce postupnosti $(u_0$, ...,$u_I)$ a $(v_0$, ...,$v_J)$. Náš postup v ďalšej časti článku ale funguje len s takýmito postupnosťami. Preto sme v tejto časti popísali len špeciálny prípad interpolácie pre takto určené uzly.
}

\section*{Nový postup generovania bikubických splajnov}

V rámci svojej bakalárskej práce som popisoval tento postup pre kubické splajnové kriovky triedy C2, teda splajny, kde interpolovaná funkcia $f$ je typu $\mathbb{R} \to \mathbb{R}$. Jedným z cieľom tejto práce je postup zovšeobecniť pre bikubické splajny, teda pre splajny, kde interpolovaná funkcia $f$ je typu $\mathbb{R} \times \mathbb{R} \to \mathbb{R}$.

Nižšie popíšeme redukovaný systém podľa doc. Töröka používaný pre kubické splajny. Neznáme $D_0$, ..., $D_{K+1}$ sú hľadané hodnoty prvých derivácií a $Y_0$, ..., $Y_{K+1}$, v prípade kubického splajnu, predstavujú funkčné hodnoty interpolovanej funkcie $f$ v uzloch $u_0$ až $u_{K+1}$.
 
\begin{multline} \label{eq:reducedL}
\begin{pmatrix}
-14 & 1 & 0 & \hdots & 0 & 0\\
1 & -14 & 1 & \hdots & 0 & 0\\
0 & 1 & -14 & \hdots & 0 & 0\\
\vdots & \vdots & \vdots & \ddots & \vdots & \vdots\\
0 & 0 & 0 & \hdots & -14 & 1 \\
0 & 0 & 0 & \hdots & 1 & \mu\\
\end{pmatrix} 
\cdot
\begin{pmatrix}
D_2\\
D_4\\
D_6\\
\vdots \\
D_{v-2}\\
D_{v}\\
\end{pmatrix}
=\\
\begin{pmatrix}
\frac{3}{\Delta }(Y_4 - Y_0) - \frac{12}{\Delta }(Y_3 - Y_1) - D_0\\
\frac{3}{\Delta }(Y_6 - Y_2) - \frac{12}{\Delta }(Y_5 - Y_3)\\
\frac{3}{\Delta }(Y_8 - Y_4) - \frac{12}{\Delta }(Y_7 - Y_5)\\
\vdots \\
\frac{3}{\Delta }(Y_\nu - Y_{\nu-4}) - \frac{12}{\Delta }(Y_{\nu-1} - Y_{\nu-3})\\
\frac{3}{\Delta }(Y_{\nu+\tau} - Y_{\nu-2}) - \frac{12}{\Delta }(Y_{\nu+1} - Y_{\nu-1} - \theta D_{K+1})\\
\end{pmatrix}\text{,}
\end{multline}

kde 
\begin{gather} \label{eq:reducedOddD}
\begin{aligned}
%\begin{split}
\mu &= -15\text{, }\tau = 0\text{, }\theta = -4\text{, a }\nu = K \text{,}&\text{ak }K\text{ je párne,}\\
\mu &= -14\text{, }\tau = 2\text{, }\theta = 1\text{, a }\nu = K - 1 \text{,}&\text{ak }K\text{ je nepárne,}
%\end{split}
\end{aligned}
\end{gather}
Hodnoty $D_k$ pre nepárne $k$ vypočítame takto:
\begin{multline} \label{eq:reducedR}
D_k = \frac{1}{4}\left(\frac{3}{\Delta}(Y_{k+1} - Y_{k-1}) - D_{k-1} - D_{k+1}\right)\text{, } k \in \{1, 3, \dots, \nu + \tau - 1\}
\end{multline}

Pri kubických splajnoch teda potrebujume vypočítať len prvé derivácie $D_0$, ..., $D_{K+1}$ v uzloch $u_0$, ..., $u_{K+1}$. 
Pri bikubických povrchoch máme na základe de Boora 4 typy sústav, kde postupne rátame parciálne derivácie $\partial x$ a $\partial y$, pričom dvomi typmi z týchto sústav vypočítame parciálne derivácie $\partial xy$. 

\section*{Zrýchlenie}

Ako pri de Boorovi tak aj pri našom spôsobe generovania derivácií dostávame niekoľko systémov trojdiagonálnych lineárnych rovníc. Štandardný spôsob riešenia takýchto rovníc

\begin{equation}
\begin{pmatrix}
b & 1 & 0 & \hdots & 0 & 0\\
1 & b & 1 & \hdots & 0 & 0\\
0 & 1 & b & \hdots & 0 & 0\\
\vdots & \vdots & \vdots & \ddots & \vdots & \vdots\\
0 & 0 & 0 & \hdots & b & 1 \\
0 & 0 & 0 & \hdots & 1 & b\\
\end{pmatrix} 
\cdot
\begin{pmatrix}
d_1\\
d_2\\
d_3\\
\vdots \\
d_{K-1}\\
d_{K}\\
\end{pmatrix}
=
\begin{pmatrix}
r1 - d_0\\
r2\\
r3\\
\vdots \\
r_{K-1}\\
r_K - d_{K+1}\\
\end{pmatrix}\text{,}
\end{equation}

spočíva v $LU$ dekompozícii $A\mathbf{d} = L\underbrace{U\mathbf{d}}_{\mathbf{y}} = \mathbf{r}$:

\begin{equation}
A = 
\begin{pmatrix}
1 & 0 & 0 & \hdots & 0 & 0\\
\lambda_2 & 1 & 0 & \hdots & 0 & 0\\
0 & \lambda_3 & 1 & \hdots & 0 & 0\\
\vdots & \vdots & \vdots & \ddots & \vdots & \vdots\\
0 & 0 & 0 & \hdots & 1 & 0 \\
0 & 0 & 0 & \hdots & \lambda_{K} & 1\\
\end{pmatrix} \text{,}
\begin{pmatrix}
\upsilon_1 & 1 & 0 & \hdots & 0 & 0\\
0 & \upsilon_1 & 1 & \hdots & 0 & 0\\
0 & 0 & \upsilon_2 & \hdots & 0 & 0\\
\vdots & \vdots & \vdots & \ddots & \vdots & \vdots\\
0 & 0 & 0 & \hdots & \upsilon_{K-1} & 0 \\
0 & 0 & 0 & \hdots & 0 & \upsilon_K\\
\end{pmatrix}\text{,}
\end{equation}
\newline
Pre $k$ z $\{2, \dots, K\}$ sú hodnoty $\upsilon_k$ a $\lambda_k$ určené takto:

\begin{equation} \label{eq:LU}
\upsilon_k = b \text{,} \left\{ \lambda_k = \frac{1}{\upsilon_{k-1}}, \upsilon = b - \lambda_k\right\} \text{,} k \in \{2, \dots, K\}\text{.}
\end{equation}

Pre priamy a spätný chod máme
\begin{equation} \label{eq:FwLy}
\text{Priamy: } L\mathbf{y} = \mathbf{r} \text{, } y_1 = r_1 \text{, } \left\{ y_k = r_k - \lambda_k\right\} \text{, } k \in \{2, \dots, K\}\text{,}
\end{equation}
\begin{equation} \label{eq:BwUd}
\text{Spätný: } U\mathbf{d} = \mathbf{y} \text{, } d_K = \frac{y_k}{u_k} \text{, } \left\{ d_k = \frac{1}{u_k}(y_k - d_{k+1})\right\} \text{,} k \in \{K-1, \dots, 1\}\text{.}
\end{equation}

LU dekompozíciou sa rieši ako de Boorova sústava (\ref{eq:deboorM}), tak aj naša redukovaná (\ref{eq:reducedL}).

Porovnajme si počet operácií násobenia pri oboch postupoch. Podotýkam, že toto konkrétne porovnanie sa teraz týka len pri kubických splajnoch. LU dekompozícia a spätný prechod obsahujú operáciu delenia, ktorá je v tabuľke nižšie označená $\gamma$. Symbol $\gamma$ udáva, že jedna operácia delenia má cenu práve $\gamma$ násobení. Reálna hodnota závisí od procesora, pričom býva rovná približne 3.

Môžeme vidieť, že výpočet neznámych redukovaným algoritmom je rýchlejší než de Boorov.
Zrýchlenie je $\frac{4+4\gamma}{5+2\gamma} \approx 1{,}45$, pričom hodnoty $d_k$, pre nepárne $k$ (\ref{eq:reducedOddD}) môžu byť počítané paralelne.

\begin{table}[h] 
	\resizebox{\textwidth}{!}{%
		\begin{tabular}{llllllll}
			& (\ref{eq:LU})  & (\ref{eq:deboorM}), (\ref{eq:reducedL}) & (\ref{eq:FwLy}) & (\ref{eq:BwUd}) & (\ref{eq:reducedR})  &  &  \\ \hline
			\multicolumn{1}{|l|}{{\bf Spôsob}} & \multicolumn{1}{l|}{{\bf LU}} & \multicolumn{1}{l|}{{\bf Matica}} & \multicolumn{1}{l|}{{\bf LU priamy}} & \multicolumn{1}{l|}{{\bf LU spätný}} & \multicolumn{1}{l|}{{\bf Zbytok}} & \multicolumn{1}{l|}{{\bf Celkovo}} & \multicolumn{1}{l|}{{\bf Zrýchlenie}} \\ \hline
			de Boor & $\gamma (K-1)$ & $1+K$ & $K-1$ & $\gamma K$ &  & $2(1+\gamma)K-\gamma$ & \\
			Redukovaný & $\gamma \left(\frac{K}{2}-1\right)$ & $2+2\frac{K}{2}$ & $\frac{K}{2}-1$ & $\gamma \frac{K}{2}$ & $1+2\frac{K}{2}$ & $\left(\frac{5}{2}+\gamma\right) K+2-\gamma$ & $\frac{4+4\gamma}{5+2\gamma}$ 
		\end{tabular}
	}
	\caption{Počet násobení s zrýchlenie pre K párne}
\end{table}


\section*{Implementácia}

Naším cieľom je vytvorenie webovej aplikácie na báze Microsoft Silverlight(ďalej len Silverlight). Silverlight je bezplatný nástroj na tvorbu webových aplikácií spustiteľných priamo vo webovom prehliadači. 
Samotný vývoj aplikácii môže prebiehať v ľubovoľnom programovacom jazyku bežiacom pod virtuálnym strojom Common Language Runtime. 
V našom prípade sme siahli po jazyku C\#. 

\begin{center}
	\includegraphics[scale=0.2]{silverlight.jpg}
\end{center}

Silverlight je kompatibilný s väčšinou moderných webových prehliadačov a operačných systémov včítane Microsoft Windows, Apple OS X a vďaka technológii Moonlight -- open-source implementácii Silverlight-u aj na väčšine Linuxových distribúcií.
Framework je možné použiť aj na vývoj off-line aplikácií v operačnom systéme Windows Phone, Windows 8 a Windows RT. 

\pozn{
	Naša aplikácia používa technológie, ktoré nie sú s Moonlight kompatibilné. V súčasnej dobe teda podporujeme iba MS Windows a Apple OS X.
}


Silverlight ponúka hardvérovo akcelerované API podporujúce tvorbu 3D grafických aplikácií a hier. V dobe písania článku je grafický engine a funkčný návrh užívateľského rozhrania hotový. Pracujem teraz na samotnej funkcionalite, ako je vizuálne odlíšenie rozdielov medzi splajnami, benchmarku generovania splajnov a rôznych dodatočných funkcionalít. 


\obrazok{apka.png}{0.7}{Ukážka z aplikácie}

\section*{Záver}

Ciele tejto práce sme spolu so školiteľom splnili, nový postup sa ukázal funkčný a rýchlejší ako predchádzajúce spôsoby. V súčasnosti je webová aplikácia takmer hotová, implementovali sme všetky postupy z tejto práce pričom sa dolaďujú rôzne detaily a vylepšuje ovládanie užívateľského rozhrania. Predpokladáme, že finálna verzia by mala byť hotová v priebehu mája 2014. 



%
%\bibliography{dp} %berie sa z dp.bib

%\renewcommand{\bibname}{Zoznam pou?itej literat?ry}
\begin{thebibliography}{9}
\bibitem{1}  David Salomon, Curves and Surfaces for Computer Graphics, Springer, 2006
\bibitem{2}  I. Szabó, L. Miňo, C. Török. Biquarticpolynomials in bicubicspline construction, PF UPJŠ, 2014 
\bibitem{3}  L. Miňo: Parametrické modelovanie dát komplexnej štruktúry, PF UPJŠ, 2014
\bibitem{4}  C. de Boor:Bicubicspline interpolation, Journal of Mathematics and Physics, 41(3),1962, 212-218.
\bibitem{5}  J. Albahari, B.Albahari: C\# 5.0 in a Nutshell, O'Reilly, 2012
\end{thebibliography}
%
\end{document}